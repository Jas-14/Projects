\documentclass{beamer}
\usetheme{Berkeley}
\usecolortheme{seagull}
\useoutertheme{infolines}
\useinnertheme{circles}


\usepackage[utf8]{inputenc}
\usepackage[T1]{fontenc}
\usepackage[french]{babel}
\usepackage{tikz}

\setbeamertemplate{navigation symbols}{}% pour enlever les symboles de navigation en bas de chaque page
\setbeamertemplate{footline}[frame number]%pour changer le pied de page par un affichage des numéros de slide
 
\title[CoreWar]{Travail Personnel Approfondi: CoreWar}
\author[]{Chaid Akacem Jasmine \and Baudin Hugo}
\institute{Université de Caen}
\date{\today}

\begin{document} 
\maketitle 

\section*{Introduction} %H

\begin{frame}
%Sujet et présentation de notre réalisation
\begin{itemize}
    \item CoreWar
    \item Développer une machine virtuelle
    \item Construire des programmes performants
\end{itemize}
\end{frame}

\tableofcontents %H

\section{Lecture d'un programme en RedCode}
\subsection{Reader et Parser}
\begin{frame} %J
\begin{center}
{\Large Deux classes pour effectuer la lecture}
\end{center}
\begin{columns}
\column{0.5\textwidth}
Classe \texttt{Reader}
\begin{itemize}
    \item Lecture basique d'un fichier texte
    \item Pré-traitement
\end{itemize}
\column{0.5\textwidth}
Classe \texttt{Parser}
\begin{itemize}
    \item Création d'un programme exécutable
    \item Découpage d'une ligne
    \item Vérification
    \item Conversion
\end{itemize}
\end{columns}
\end{frame}

\subsection{Traitement}
\begin{frame} %J
\begin{itemize}
    \item Découpage de la ligne
    \item Traitement différencié selon opérateur
\end{itemize}
\begin{figure}
    \centering
\begin{tikzpicture}
\node[draw=blue] (op) at (0,0) {OP};
\node[draw=red] (modifierA) at (1,0) {M$_{A}$};
\node[draw=red] (A) at (1.7,0) {A};
\node[draw=teal] (modifierB) at (2.7,0) {M$_{B}$};
\node[draw=teal] (B) at (3.4,0) {B};
\end{tikzpicture}
\caption{Schéma général d'une ligne RedCode}
    \label{lineRed}
\end{figure}
\begin{exampleblock}{Exemple traitement différencié}
La ligne \texttt{ADD \#0 1} est orienté vers le traitement par défaut.

La ligne \texttt{JMP 5} est orienté vers le traitement pour les opérateurs à un champ.
\end{exampleblock}
\end{frame}

\subsection{L'objet Instruction}
\begin{frame}%H
%Vérification + Conversion en objet Instruction (description de cet objet)
\begin{center}
    Les instructions
\end{center}
\begin{itemize}
    \item Instruction : Classe abstraite
    \item Trois méthodes principales
\end{itemize}
\begin{figure}
    \centering
    \begin{tikzpicture}
    \node[draw=black] (dat) at (0,0) {DAT};
    \node[draw=black] (mov) at (1.2,0) {MOV};
    \node[draw=black] (jmp) at (2.4,0) {JMP};
    \node[draw=black] (jmz) at (3.6,0) {JMZ};
    \node[draw=black] (add) at (4.8,0) {ADD};
    \node[draw=black] (sub) at (6,0) {SUB};
    \node[draw=black] (djz) at (7.2,0) {DJZ};
    \node[draw=black] (cmp) at (8.4,0) {CMP};
    \node[draw=black] (ins) at (4,3) {Instruction};
    
    \draw[->] (dat) -- (ins);
    \draw[->] (mov) -- (ins);
    \draw[->] (jmp) -- (ins);
    \draw[->] (jmz) -- (ins);
    \draw[->] (add) -- (ins);
    \draw[->] (sub) -- (ins);
    \draw[->] (djz) -- (ins);
    \draw[->] (cmp) -- (ins);
    \end{tikzpicture}
    \caption{Héritage de l'instruction}
    \label{fig:instruction}
\end{figure}
\end{frame}

\subsection{Les erreurs}
\begin{frame}%J
\begin{center}
Une gestion des erreurs pendant la lecture
\end{center}
\begin{columns}
\column{0.35\textwidth}
Erreurs de syntaxe
\begin{itemize}
    \item Ligne non conforme
    \item Opérateur/Mode d'adressage non reconnu
    \item Adresse qui n'est pas un nombre
\end{itemize}
\column{0.65\textwidth}
Instructions non exécutables
\begin{itemize}
    \item Ligne syntaxiquement correcte
    \item Nécessite qu'un objet Instruction soit instancié
    \item Bloquée avant son exécution
\end{itemize}
\end{columns}
\begin{exampleblock}{Exemple Instruction non exécutable}
La ligne \texttt{ADD 0 1} n'est pas exécutable

La ligne \texttt{DAT 0} n'est pas exécutable (Raison de \emph{cohérence})
\end{exampleblock}
\end{frame}

\section{CoreWar}

\subsection{Classes principales}
\begin{frame}%J
\begin{center}
{\Large La partie CoreWar: deux objets principaux}
\end{center}
\begin{columns}
\column{0.55\textwidth}
La classe \texttt{VirtualMachine}
\begin{itemize}
    \item Machine virtuelle du logiciel
    \item Exécution
    \item Stockage
    \item Insertion des programmes
\end{itemize}
\column{0.45\textwidth}
La classe \texttt{Warrior}
\begin{itemize}
    \item Un programme à exécuter
\end{itemize}
\end{columns}
\end{frame}

\subsection{Début du jeu}
\begin{frame}%J
\begin{center}
Deux initialisations de la mémoire
\end{center}
\begin{itemize}
    \item Avec des données (Ligne \texttt{DAT \#0})
    \item Avec les deux programmes qui vont se combattre
    \begin{enumerate}
        \item Placement du pointeur du premier programme
        \item Placement du premier programme
        \item Placement du pointeur du second programme selon les contraintes
        \item Placement du second programme
    \end{enumerate}
\end{itemize}
\begin{figure}
    \centering
    \begin{tikzpicture}
    \node (p1d) at (0,0) {Début P1};
    \node (p1f) at (6,0) {Fin P1};
    \node (pp1) at (0,1.5) {Pointeur P1};
    
    \node (nop2) at (3,0.3) {Pas de P2};
    \draw[dashed, red] (p1d) -- (p1f);
    \draw[->] (pp1) -- (p1d);
    \end{tikzpicture}
    \caption{Contraintes de placement}
    \label{fig:my_label}
\end{figure}
\end{frame}

\subsection{La boucle du jeu}
\begin{frame}%H
\begin{overprint}
\onslide<1>
\begin{figure}
    \centering
    \begin{tikzpicture}
    \node[draw=red] (pointeur) at (0,3) {Pointeur};
    \node[draw=blue] (instru1) at (3,3) {0 : \texttt{MOV 0 1}};
    \node[draw=black] at (3,2) {1 : Instruction B};
    \node[draw=black] at (3,1) {2 : Instruction C};
    \node[draw=black] at (3,0) {3 : Instruction D};
    \draw[->] (pointeur) -- (instru1);
    \end{tikzpicture}
    \caption{Boucle de jeu}
    \label{fig:Boucle 1}
\end{figure}

\onslide<2>
\begin{figure}
    \centering
    \begin{tikzpicture}
    \node[draw=red] (pointeur) at (0,3) {Pointeur};
    \node[draw=blue] (instru1) at (3,3) {0 : \texttt{MOV 0 1}};
    \node[draw=blue] at (3,2) {1 : \texttt{MOV 0 1}};
    \node[draw=black] at (3,1) {2 : Instruction C};
    \node[draw=black] at (3,0) {3 : Instruction D};
    \draw[->] (pointeur) -- (instru1);
    \end{tikzpicture}
    \caption{Boucle de jeu}
    \label{fig:Boucle 2}
\end{figure}

\onslide<3>
\begin{figure}
    \centering
    \begin{tikzpicture}
    \node[draw=red] (pointeur) at (0,2) {Pointeur};
    \node[draw=blue] at (3,3) {0 : \texttt{MOV 0 1}};
    \node[draw=blue] (instru2) at (3,2) {1 : \texttt{MOV 0 1}};
    \node[draw=black] at (3,1) {2 : Instruction C};
    \node[draw=black] at (3,0) {3 : Instruction D};
    \draw[->] (pointeur) -- (instru2);
    \end{tikzpicture}
    \caption{Boucle de jeu}
    \label{fig:Boucle 3}
\end{figure}

\onslide<4>
\begin{figure}
    \centering
    \begin{tikzpicture}
    \node[draw=red] (pointeur) at (0,2) {Pointeur};
    \node[draw=blue] at (3,3) {0 : \texttt{JMP 1}};
    \node[draw=blue] (instru2) at (3,2) {1 : \texttt{DAT \#0}};
    \node[draw=black] at (3,1) {2 : Instruction C};
    \node[draw=black] at (3,0) {3 : Instruction D};
    \draw[->] (pointeur) -- (instru2);
    \end{tikzpicture}
    \caption{Boucle de jeu}
    \label{fig:Boucle Fin}
\end{figure}
\end{overprint}

\end{frame}

\section{Génération de programme}

\subsection{L'algorithme génétique}
\begin{frame}%H
\begin{columns}
\column{0.4\textwidth}
\begin{itemize}
    \item Créer un programme performant
    \item Analogue à la biologie
    \begin{itemize}
        \item Sélection
        \item Mutations
        \item Croisement des individus sélectionnés
    \end{itemize}
    \item Début aléatoire
\end{itemize}
\column{0.6\textwidth}
\begin{figure}
    \centering
    \begin{tikzpicture}
        \node[draw=black] (GA) at (0,4) {Génération aléatoire};
        
        \node[draw=black] (C) at (0,3) {Combats};
        
        \node[draw=black] (S) at (0,2) {Sélection};
        
        \node[draw=black] (M) at (0,1) {Mutations};
        
        \node[draw=black] (Cr) at (0,0) {Croisements};
        
        \draw[->] (GA) -- (C);
        \draw[->] (C) -- (S);
        \draw[->] (S) -- (M);
        \draw[->] (M) -- (Cr);
        
        \draw[bend right,->,red] (1.1,0) to node [auto,swap]{Boucle} (0.9,3);
    \end{tikzpicture}
    \caption{Déroulement de l'algorithme}
    \label{fig:GP}
\end{figure}
\end{columns}
\end{frame}

\subsection{Combat}
\begin{frame}%J
\begin{center}
{\Large Partie Combat}
\begin{itemize}
    \item Méthode \texttt{fight}
    \item Adaptation au problème
    \item Statistiques sur les Warriors
\end{itemize}
\vspace{5mm}
\begin{exampleblock}{Notre implémentation: \emph{Tournoi}}
\begin{itemize}
    \item Complexité raisonnable
    \item Les meilleurs confrontés aux meilleurs
\end{itemize}
\end{exampleblock}
\end{center}
\end{frame}

\subsection{Sélection}
\begin{frame}
\begin{center}
{\Large Partie Sélection}
\begin{itemize}
    \item Méthode \texttt{doSelection}
    \item Garder la meilleure partie de la population
    \item A partir des statistiques générées
\end{itemize}
\vspace{5mm}
\begin{exampleblock}{Notre implémentation: \emph{Les 20 meilleurs}}
\begin{itemize}
    \item Adapté à notre implémentation du combat
    \item Complexité d'un algorithme de tri
\end{itemize}
\end{exampleblock}
\end{center}
\end{frame}

\subsection{Evolution}
\begin{frame}%J
\begin{center}
{\Large Partie Evolution}
\begin{itemize}
    \item Méthode \texttt{mutAllWarriors}
    \item Outils de base
    \item Réutilisation avec un seul outil, muni des probabilités
    \item Intervient sur les programmes
\end{itemize}
\begin{exampleblock}{Notre implémentation}
Outils de base
\begin{itemize}
    \item Remplacement d'une ligne
    \item Délétion d'une ligne
    \item Rajout d'une ligne
\end{itemize}
Le tout encapsulé dans une classe qui va utiliser les trois outils successivement.
\end{exampleblock}
\end{center}
\end{frame}

\subsection{Croisement}
\begin{frame}%H
\begin{center}
{\Large Partie Croisement}
\end{center}
\begin{itemize}
    \item Méthode \texttt{cross} et \texttt{crossAll}
    \item Nombre de croisements définis
    \item Prend les instructions des meilleurs programmes
    \item Créé de nouveaux programmes
\end{itemize}
\begin{exampleblock}{Notre implémentation}
\begin{itemize}
    \item Sélection aléatoire de deux programmes
    \item Sélection aléatoire de ligne
\end{itemize}
\end{exampleblock}
\end{frame}

\section{Conclusion}%J

\begin{frame}
\begin{center}
{\Large Conclusion}
\vspace{5mm}
\begin{itemize}
    \item Un CoreWar complétement orienté objet
    \item Une gestion des erreurs avant exécution
    \item Une génération de programme modulable selon les besoins
\end{itemize}
\end{center}
\end{frame}

\end{document}